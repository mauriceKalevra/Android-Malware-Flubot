% class definitions
\documentclass[a4paper,12pt]{scrartcl}

% Packages

\usepackage[utf8]{inputenc}
\usepackage[ngerman]{babel}
\usepackage[T1]{fontenc}
\usepackage{graphicx}
\usepackage{lmodern}
\usepackage{tabto}
\usepackage{listings}
\usepackage[backend=biber, style=authoryear]{biblatex}


% Front page

\title{ Flubot:\\ Android-Malware verbreitet sich über Fake-Patches }
\subtitle{Cybersecurity}
\author{Moritz Rupp}
\date{Wintersemester 2021/22}


% Mathepakete
\usepackage{amsfonts}
\usepackage{amsmath}

%Document start 
\begin{document}
 


\maketitle
\newpage
\tableofcontents
\newpage


\section{Abstract}
Die Android-Malware 'Flubot' trat das erste mal Ende des Jahres 2020 auf. Anfangs \\tausende, später Millionen von Andriod Nutzern berichteten über eine vielzahl von verdächtigen SMS Nachrichten. Zwar unterschieden sich die Mitteilungen in gewissen Details, jedoch war die Kernaussage immer die gleiche! Ein Tracking code mitsamt Link sollte auf ein Packet aufmerksam machen das in kürze bei einem eintreffen würde! Als Absender wurden oftmals Amazon und Ebay genannt, wobei in einzelfällen auch weitere kleinere Online Kaufhäuser genannt wurden. Für It Affine Menschen war schnell klar das dass ganze eine groß Angelegte Phising Kampagne darstellte! Der Angriff führte zu Schaden in x Höhe..\\
So viele Geräte wurden infiziert\dots.. 



Androit Malware kann sich auf viele verschiedene Arten verbreiten. Sei es über\\ komprimierte App installationen, Downloads oder phishing Angriffe! Letzteres wurde Anfang 2021 genutzt um Hunderttausende Geräte zu Infizieren bzw. Abzugreifen!\\
Der Angriff traf tausende Geräte und führte zu einem neuen Bewustsein von Phising Angriffen!  

\newpage
\section{Einführung}
\subsection{Was ist Flubot}
Gegen Ende des Jahres 2020 erschienen erstmals Meldungen über eine neue Android Malware namens 'Flubot'.


\section{Funktionsweiße}
\dots
\section{Technische Analyse}

\dots
\section{Lösungen für Phishing Angriffe}
\dots
\section{Ausblick und Conclusion}
\end{document}

