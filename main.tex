% class definitions
\documentclass[a4paper,12pt]{scrartcl}

% Packages

\usepackage[utf8]{inputenc}
\usepackage[ngerman]{babel}
\usepackage[T1]{fontenc}
\usepackage{graphicx}
\usepackage{lmodern}
\usepackage{tabto}
\usepackage{listings}
\usepackage{quoting} %
\usepackage{lipsum}
\quotingsetup{font={itshape}, leftmargin=2em, rightmargin=0in, vskip=1ex}

\usepackage[backend=biber, style=authoryear]{biblatex}


% Front page

\title{ Flubot:\\ Android-Malware verbreitet sich über Fake-Patches }
\subtitle{Cybersecurity}
\author{Moritz Rupp}
\date{Wintersemester 2021/22}


% Mathepakete
\usepackage{amsfonts}
\usepackage{amsmath}

%Document start 
\begin{document}
 


\maketitle
\newpage
\tableofcontents
\newpage


\section{Abstract}
Die tägliche Nutzung von Smartphones ist nunmehr seit über 10 Jahren in der breiten Geselschaft angekommen. Egal ob private oder geschäftliche  kurznachrichtendienste, Online Banking oder shopping. In allen Bereichen des sozialen und geschäftlichen Lebens finder das smartphone Anwendung. Auch neuere Erscheinungen wie der online handel mit crypto währungen findet breit über mobile endgeräte statt. \\
Derzeit sind laut der Musterquelle über 3 Milliarden mobile Geräte im Umlauf. Davon nutzt über die Hälfte das Betriebssystem Android.
Die große Anzahl an ähnlichen Geräten mit einer ähnlichen Software auf denen sensitive Daten laufen bietet eine Menge an Angriffsmöglichkeiten.
Phishing gilt seit jahren als eine der größten gefahren.
Meist durch imitation eines Anbieters oder Dienstes wird versucht Opfer eine externe Software zu installieren. Diese liest nun je nach Angriff belieb Daten aus.
Das Botnetz Flubot ist solch ein Phishing angriff und wird in dieser Arbeit untersucht.



\newpage
\section{Einführung}
Die Android-Malware 'Flubot' trat das erste mal Ende des Jahres 2020 auf. Anfangs \\tausende, später Millionen von Android Nutzern berichteten über eine Vielzahl von verdächtigen SMS Nachrichten. Zwar unterschieden sich die Mitteilungen in gewissen Details, jedoch war die Kernaussage bzw. Intension der Nachricht immer die gleiche.  Durch einen Vorwand sollte das Opfer dazu geleitet werden auf einen Link zu klicken der zu einer Webseite führte. Abhängig von Inhalt der Nachricht sollten dort verschiedene Dienste Angeboten werden. Anfangs stellte der Köder eine vermeintlich verpasste Voicemail da, welche nur durch eine zu installierende Applikation abgehört werden könne. Im weiteren Verlauf der Angriffe wurden zudem Packetlieferdienste imitiert die auf ein bald eintreffendes Packet aufmerksam machen sollten. 
Seit Mitte des Jahres 2021 wird nun versucht auf gestolene Privataufnahmen hinzuweißen die auf einem Portal hochgeladen wurden. In allen Fällen enthielten die Nachrichten einen Link welcher zu einer komprementierten Webseite führte. Diese enthielt je nach Szenario die Aufforderung eine APK herunterzuladen welche es möglich mache den jeweiligen Dienst zu Nutzen.\\
Durch Installation dieser Software infizierte man sein Gerät und wurde Teil des Botnetzes. In folge dessen durchläuft die Malware das Adressbuch und verbreitet sich daraufhin namesgebend wie ein Flobefall über private Kontakte!\\
Für Branchenkenner war schnell klar das dass ganze eine groß Angelegte Phising Kampagne darstellte!\\ Aufgrund der Tatsache das Flubot nach wie vor im Umlauf ist lässt sich schwer einschätzen wie viele Geräte derzeit Infiziert sind. Im Deutschen Raum werden Schätzungen zwischen 1-3 Millionen Endgeräten laut. Auch finanziell ist die Schadenssumme derzeit nicht präzise zu bestimmen. Schätzen sprechen hier jedoch zwischen 4-8 Millarden Euro.\\
Durch diese Ausmaße führte Flubot zu einem neuen Bewustsein von Phising Angriffen.
Gegenstand dieser Arbeit ist es die Bedrohungslage von solch Angriffen zu verstehen und Lösungsansätze zu finden.\\
Dafür wird anfangs die Funktionsweiße genauer untersucht und erleutert.
Anschließend wird durch eine Technische Analyse die eigentliche Anwendung hinter Flubot aus Angreifersicht beleuchtet.\\
Darauf aufbauend werden nun verschiedene Lösungsansätze diskutiert und vorgeschlagen.
Abschließend werden die neu erlernten Kenntisse zusammengeführt und ein Ausblick in zukünftige Bedrohungslagen und Lösungen gewagt.
																					
\newpage
\section{Funktionsweiße}
\subsection{Verbreitung}
Flubot lässt sich terminologisch als 'Banking Trojaner' einordnen. Das heißt Hauptintension der Schadware besteht darin Konto information abzugreifen um daraus finanziellen profil zu erlangen. Die Infektion und Verbreitung von Trojaner werden häufig mithilfe von Botnetzen betrieben. Diese bestehen aus oftmals tausenden Geräten die automatisiert die Malware an sich betrieben und sich zudem über das befallene Gerät weiter verbreiten.
Im Fall von Flubot wird davon ausgegangen das ein großteil der ersten Mobilfunknummern durch einen Datenleak von Facebook stammen. Mitte  
2020 war es Angreifern gelungen persönliche Daten von 11 Millionen Britischen Facebook accounts abzugreifen. Des weiteren wurden höchstwahrscheinlich weitere Datenleaks der letzten Jahre für die verbreitung genutzt!\\
Durch die Länder Vorwahl ist Flubot in der Lage aus einer Liste von Phishing Ködern zu wählen die zu Sprache und Region des Opfers passen.
Eine Phishing SMS setzte sich anfangs lediglich aus einer kurzen Nachricht und einem Link zusammen.
\newline
\begin{quoting} 
 voice message received:\\
 hxxp://tantawy-group[.com/z.php?REDACTED
\end{quoting}



Nachdem jedoch viele dieser Nachrichten durch SMS-Filter seitens der Mobilfunkanbieter geblockt wurden, passten sich die Angreifer durch komplexere nachrichten an. Nun wurde ein prefix aus zufälligen Zahlen und Buchstaben vor der eigentlich Nachricht eingefügt. Auch im Nachrichtentext wurden teilweise einzelne Buchstaben geflipt.
Im Verlauf der Angriffe wechselten die Köder Nachrichten sehr häufig. So wurde vorallem im deutschsprachigen Raum  mit Packet tracking nachrichten gearbeitet. Ähnliche Imitationen sind Fedex und UPS für den Amerikanischen Raum. Zuletzt wurden zudem Sicherheitsupdates gegen Flubot selbs verwendet.

\newpage
\section{Technische Analyse}

\dots
\newpage
\section{Lösungen für Phishing Angriffe}
\dots
\newpage
\section{Ausblick und Conclusion}

\end{document}

