% class definitions
\documentclass[a4paper,12pt]{scrartcl}

% Packages

\usepackage[utf8]{inputenc}
\usepackage[ngerman]{babel}
\usepackage[T1]{fontenc}
\usepackage{graphicx}
\usepackage{lmodern}
\usepackage{tabto}
\usepackage{listings}
\usepackage[backend=biber, style=authoryear]{biblatex}


% Front page

\title{ Flubot:\\ Android-Malware verbreitet sich über Fake-Patches }
\subtitle{Cybersecurity}
\author{Moritz Rupp}
\date{Wintersemester 2021/22}


% Mathepakete
\usepackage{amsfonts}
\usepackage{amsmath}

%Document start 
\begin{document}
 


\maketitle
\newpage
\tableofcontents
\newpage


\section{Abstract}



\newpage
\section{Einführung}
Die Android-Malware 'Flubot' trat das erste mal Ende des Jahres 2020 auf. Anfangs \\tausende, später Millionen von Android Nutzern berichteten über eine Vielzahl von verdächtigen SMS Nachrichten. Zwar unterschieden sich die Mitteilungen in gewissen Details, jedoch war die Kernaussage bzw. Intension der Nachricht immer die gleiche.  Durch einen Vorwand sollte das Opfer dazu geleitet werden auf einen Link zu klicken der zu einer Webseite führte. Abhängig von Inhalt der Nachricht sollten dort verschiedene Dienste Angeboten werden. Anfangs stellte der Köder eine vermeintlich verpasste Voicemail da, welche nur durch eine zu installierende Applikation abgehört werden könne. Im weiteren Verlauf der Angriffe wurden zudem Packetlieferdienste imitiert die auf ein bald eintreffendes Packet aufmerksam machen sollten. 
Seit Mitte des Jahres 2021 wird nun versucht auf gestolene Privataufnahmen hinzuweißen die auf einem Portal hochgeladen wurden. In allen Fällen enthielten die Nachrichten einen Link welcher zu einer komprementierten Webseite führte. Diese enthielt je nach Szenario die Aufforderung eine APK herunterzuladen welche es möglich mache den jeweiligen Dienst zu Nutzen.\\
Durch Installation dieser Software infizierte man sein Gerät und wurde Teil des Botnetzes. In folge dessen durchläuft die Malware das Adressbuch und verbreitet sich daraufhin namesgebend wie ein Flobefall über private Kontakte!\\
Für Branchenkenner war schnell klar das dass ganze eine groß Angelegte Phising Kampagne darstellte!\\ Aufgrund der Tatsache das Flubot nach wie vor im Umlauf ist lässt sich schwer einschätzen wie viele Geräte derzeit Infiziert sind. Im Deutschen Raum werden Schätzungen zwischen 1-3 Millionen Endgeräten laut. Auch finanziell ist die Schadenssumme derzeit nicht präzise zu bestimmen. Schätzen sprechen hier jedoch zwischen 4-8 Millarden Euro.\\
Durch diese Ausmaße führte Flubot zu einem neuen Bewustsein von Phising Angriffen.
Gegenstand dieser Arbeit ist es die Bedrohungslage von solch Angriffen zu verstehen und Lösungsansätze zu finden.\\
Dafür wird anfangs die Funktionsweiße genauer untersucht und erleutert.
Anschließend wird durch eine Technische Analyse die eigentliche Anwendung hinter Flubot aus Angreifersicht beleuchtet.\\
Darauf aufbauend werden verschiedene Lösungsansätze diskutiert und vorgeschlagen.
Abschließend werden die neu erlernten Kenntisse zusammengeführt und ein Ausblick in zukünftig  Szenarien gewagt.
																					
\newpage
\section{Funktionsweiße}

\section{Technische Analyse}

\dots
\section{Lösungen für Phishing Angriffe}
\dots
\section{Ausblick und Conclusion}
\end{document}

