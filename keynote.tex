\documentclass{beamer}

%style
\mode<presentation>
\usetheme{Boadilla}

%Packages
\usepackage[utf8]{inputenc}
\usepackage[ngerman]{babel}
\usepackage{graphicx}
\usepackage{booktabs}
\usepackage{mathtools}
\usepackage{amsmath}
\usepackage{listings}
\usepackage[utf8]{inputenc}
\usepackage[ngerman]{babel}
\usepackage[T1]{fontenc}
\usepackage{lmodern}
\usepackage{tabto}
\usepackage{listings}
%bibtex
\usepackage[backend=biber, style=authoryear]{biblatex}
\addbibresource{referenzen.bib}

%Einstellungen der Präsentation
\title[Cybersecurity]{Flubot:\\ Android-Malware verbreitet sich über Fake-Patches}
\author{Moritz Rupp}
\institute[MR]{Hochschule Albstadt-Sigmaringen}

\date{15. Januar - WS 21/22}

%Beginn der Präsentation
\begin{document}

%Titelseite
\begin{frame}
\titlepage
\end{frame}
%Inhaltsverzeichnis
\begin{frame}
\frametitle{Inhalt}
\tableofcontents    
\end{frame}
\begin{frame}{Trivia}
 \section{Trivia}
 - Flubot ist ein Banking Trojaner\\
 - Erstes Auftreten Ende 2020
 - 
\end{frame}
\begin{frame}{Funktionsweise}
 Verteilung... \\
 Infection ... \\
\end{frame}
\begin{frame}{Technische Analyse}
In Java geschrieben

\end{frame}
\end{document}
